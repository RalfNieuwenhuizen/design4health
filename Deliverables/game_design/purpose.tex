%1. serious game (purpose and reality)
%- purpose of the game; what does it need to bring about beyond the context of the game?
%- what is the strategy you chose to achieve that purpose?; how does it work, and why?
%- what is the role of the main operations and game elements in that strategy? how are these elements kept in balance?
%- describe the model you have developed for your simulation, and explain the choices made

The game is designed to be fun and engaging for users, a necessary characteristic to reach the actual purpose of the game: \textit{Stimulating a healthy lifestyle}. 

\subsection{Real life implications}
From a real life point of view, the user is encouraged by the game to perform isolated exercises that are carefully selected to cover all body regions. The user is also able to set personal goals by choosing to perform sets of exercises focused on a specific body region. All the real life implications of performing exercises are explicitly mentioned in the game, while being seamlessly incorporated in the storyline. Finally, the game does stimulate the user to think about the current lifestyle.

\subsection{Way of implementation}
Different game elements help to include the real life implications in the game and communicate them to the user. To be able to provide specific feedback, the body is categorized in $5$ specific body parts: The back, the chest, the abs, the arms and the legs. \\
\\
The exercises are specifically selected to train the different body parts to support the user's general fitness. The challenges include specific sets of exercises and can voluntarily be selected by the user. The user is able to focus on specific body parts by selecting challenges which contain exercises for that body part. The progression of the user is represented in the B.O.D.Y. which is showing the progress for each specific body part and can be compared to the historical progress via a statistics screen. \\
\\
To support a healthy lifestyle outside the game, the user is motivated to think about it by daily task provided by the mentor. This daily task is introduced to trigger the user to think about a healthy lifestyle while not playing the game. 

\subsubsection{Exercises}
The game contains a variety of exercises with different aims. The exercises contain information about the way they should be executed and which groups of muscles are trained. During the execution of the exercise, a smartphone is used to measure the movements and to provide acoustic feedback. The exercises are chosen in cooperation with a physiotherapist, who provided feedback on how to execute them correctly.

\subsubsection{Challenges}
Challenges combine exercises into a workout in a playful way. This is an easy way for the user to keep his whole body strong and fit, and prevent injuries. Challenges also give users the opportunity to do exercises at each time of the day, without having to wait for crops or livestock to get ready for harvest. A challenge without requirements is available as well, which lets the user walk a certain amount of steps. This is a good way to earn extra money when all is spent. This is a metaphor of \emph{``no matter how physically impaired you are, you can always take some steps to get better"}. 

\subsubsection{B.O.D.Y.}
The Bionic Outer Dimension Yeosuit (B.O.D.Y.) represents the progress of the real life body of the user. By executing the exercises the user earns experience points for the represented body parts. When sufficient experience points are earned for each body part, the B.O.D.Y.-level will increase, allowing the user to do more challenges and clone more crops and livestock. This is similar to the way a better general fitness allows someone to do more demanding tasks in real life. Every time you open the B.O.D.Y. screen you will be advised to give attention to your weakest body part.

\subsubsection{Statistics}
While the user progresses in the game, it motivates to have a look back in history on all efforts that have been made. Therefore the game reflects the user's long-term progress, encouraging to continue exercising regularly throughout the day and also to focus on all different body regions equally. 

\subsubsection{Daily tasks}
Every day, the user is challenged to do a real life task, such as taking the stairs instead of the elevator. These tasks are not obtrusive nor enforced, but are there to make the user aware of living healthy outside of the game, providing easy tasks to make everyday life more healthy. 