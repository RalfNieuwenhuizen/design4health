The user interface has undergone extensive testing. Based on the test results the interface is changed multiple time to make it more intuitive. 

\subsection{Design}
The graphics were done in Adobe Illustrator and finalized in Adobe Photoshop. We decided for cartoon-style design because of its ease of drawing, ease of implementation and high creativity potential.

\subsection{Map}
To show the farm on the map, we decided for the isometric view. Compared to a regular rectangular view, it is much more visually appealing and relates to the story -- a birds eye view. Compared to a fully 3D approach, it is an order of magnitude easier to implement, and also provides significantly higher frame rates on a smartphone. By stacking tiles with the ZigZag algorithm, we create a fake 3D perspective.

\subsection{Sounds}
Sound were used to pull the player into the game. From our testing sessions we gathered that players mostly appreciate music and sounds, but they require a way to disable them. Therefore we introduced the settings window, where sounds and music can be disabled.

\subsection{Clickable objects}

To make it more clear that a certain crop or livestock is ready and needs action, it is now glowing. This drags the immediate attention of the user.

Moreover, we increased the size of buttons significantly compared to the first playable, to decrease the time the user has to spend positioning his finger on the screen. The drawback of this action was that we had to create additional screens because the display got too cluttered.

\subsection{Daily tasks}
Daily tasks are implemented to create more effect in the normal life of the user. The game has been extended and shows the uncle on the farm. The uncle provides the user with daily tasks. This seemed to be a convenient way to show more information.

\subsection{Introduction}
At the very start it was not clear what the buttons at the bottom of the screen were used for. We decided to create an introductional session to show all the features of the game in a playful way. After some tests it seemed that the introduction was obtrusive and enforcing the user too much. This has been improved by only giving a brief introduction, followed by a list of tasks which gives the freedom to the user to explore the game on his own.