\documentclass[a4paper,11pt,notitlepage]{report}

\usepackage[utf8]{inputenc}
\usepackage{amsmath}
\usepackage{listings}
\usepackage[hidelinks]{hyperref}
\usepackage{catoptions}
\usepackage[left=0.8in, right=0.8in, top=0.8in, bottom=0.8in]{geometry}
\usepackage{color}
\usepackage{soul}
\usepackage{float}
\usepackage{framed}
\usepackage[sc]{mathpazo}
\linespread{1.20}         % Palatino needs more leading (space between lines)
\usepackage[T1]{fontenc}
\usepackage{microtype}
\usepackage{enumerate}
\usepackage{courier}
\usepackage{graphicx}
\usepackage{enumitem}
\usepackage{lipsum}
\usepackage{tikz}
\usepackage{caption}
\usepackage{subcaption}
\usepackage{verbatim}
\usetikzlibrary{shapes,arrows}

\graphicspath{ {./Images/} }

\pdfinfo{
  /Title    (Building Serious Games - Design4Health)
  /Author   (Ralf Nieuwenhuizen, David Prihoda, Ismini Psuxoula, Arnold Schutter, Shen Shuheng)
  /Creator  (Ralf Nieuwenhuizen, David Prihoda, Ismini Psuxoula, Arnold Schutter, Shen Shuheng)
  /Producer (Ralf Nieuwenhuizen, David Prihoda, Ismini Psuxoula, Arnold Schutter, Shen Shuheng)
  /Subject  (Building Serious Games)
}

% Settings for hyperref package (e.g. wat \autoref en \nameref moeten doen)
\hypersetup{
  colorlinks  = false,
  linkcolor   = [rgb]{0.1,0.1,0.5},
  citecolor   = [rgb]{0.5,0.1,0.1},
  filecolor   = [rgb]{0.1,0.5,0.5},
  urlcolor    = [rgb]{0.1,0.1,0.7}
}

\newcommand{\todo}[1] {\hl{#1}}
\setlength{\parindent}{0cm}

\begin{document}

% Define block styles
\tikzstyle{block} = [rectangle, draw, fill=white!20, 
    text width=7em, text centered, rounded corners, minimum height=4em]
\tikzstyle{blockSmall} = [rectangle, draw, fill=white!20, 
    text width=4em, text centered, rounded corners, minimum height=3em]
\tikzstyle{line} = [draw, -latex']
		
\begin{center}
\vskip 1cm
{\Huge Design4Health \vskip 2mm}
{\Large Synopsis \& Responsibilities \vskip 1cm}

{\normalsize \textbf{Ralf Nieuwenhuizen ($000000$) -- \textbf{David Prihoda ($000000$)} -- \textbf{Ismini Psuxoula ($000000$)} -- \textbf{Arnold Schutter} ($4260724$) -- \textbf{Shen Shuheng ($000000$)} \vskip 3cm}

\end{center}

%\newpage
%\tableofcontents
%\newpage

\chapter{Team}

Our team exists of $5$ members, all having an own main responsibility. 

\section{Responsibilities}

\begin{itemize}
	\item Ralf Nieuwenhuizen: Communication
	\item David Prihoda: Design architect
	\item Ismini Psuxoula: Lead programmer
	\item Arnold Schutter: Lead Game Design
	\item Shen Shuheng: Lead Testing
\end{itemize}

\subsection{Communication}
The communicator is responsible for the timely communication with external parties and the teacher including weekly updates. 

\subsection{Design Architect}
The design architect is responsible for game play and graphics. Game play (fun) should continuously be checked for and the graphics should be made according to a graphical design plan. The design architect is responsible for this plan and prioritizing tasks.

\subsection{Lead Programmer}
The lead programmer is responsible for keeping the overview of the software (architecture) and for quality. The lead programmer prioritizes the milestones for the software and checks for the quality and coherence. When deliverables are not satisfactory, the lead programmer is allowed to let deliverables be rectified.

\subsection{Lead Game Design}
The lead game designer is responsible for the overall planning and the coherence between the software and the game design. 

\subsection{Lead Testing}
The lead tester is responsible for weekly testing the deliverables for appearance, errors/bugs, quality and coherence. The Lead Tester prioritizes the tasks to be improved together with the Lead programmer. 

\section{Planning}

\subsection{Meetings}
Our weekly meetings are at:
\begin{itemize}
	\item Tuesday $12.30$ - $16.30$
	\item Thursday $13.45$ - $17.30$
	\item Friday $13.45$ - $16.30$
\end{itemize}

\subsection{Schedules}

\begin{figure}[H]
	\centering
		\includegraphics[width=1.00\textwidth]{Images/PlanningSmall.png}
	\caption{Global planning of the project}
	\label{fig:Planning}
\end{figure}

\begin{figure}[H]
	\centering
		\includegraphics[width=1.00\textwidth]{Images/Todolist.png}
	\caption{Global todo list of the project}
	\label{fig:Todolist}
\end{figure}

\chapter{Game design}
The official game description does not contain specific requirements. The purpose of the game is to motivate or help people to do their exercises, possibly for tasks provided by a physiotherapist. \\
\\
The first concept of design is an engaging game for which the user needs to gather more points to get better in the game. The points can be gathered by doing the exercises according to the training scheme, possibly provided by a physiotherapist. Every successfully completed training day is worth a specific amount of points determined by the physiotherapist. While the user is doing it's exercises properly every day, a combo is created and each training day will be worth more points over time. Skipping a training day will stop the combo. \\
\\
To control the user for doing the exercises properly, training data should be uploaded after completing the daily training. The data can be checked by a supervisor like the physiotherapist once in a while to check for correct performance of the exercise. The required data type differs per type of exercise. For specific exercises provided by the physiotherapist the user could make a video of the performance of the exercises. GPS-data would be more useful when the exercises are running, cycling or walking. \\
\\
Several engaging games should be available in a game platform, specific for different age categories and gender, from which a user can pick one game. The games in the game platform should work independent of the other part of the system. The games should however facilitate the input of gathered points directly in the game. \\
\\
The construct of the system is shown in \ref{fig:gamedesignscheme}. 

\begin{figure}[H]
	\centering
		\includegraphics[width=0.80\textwidth]{Images/gamedesignscheme.jpg}
	\caption{Scheme of the game design}
	\label{fig:gamedesignscheme}
\end{figure}



%\bibliography{synopsis}
%\bibliographystyle{plain}

\end{document}
